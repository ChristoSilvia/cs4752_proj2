\documentclass{article}

\usepackage{amsmath}
\usepackage{amsfonts}

\DeclareMathOperator{\tr}{tr}

\begin{document}

\title{Summary of Secondary Objective Function}
\author{Christopher Silvia, Isaac Quereshi, Zachary Vineager}

\maketitle

\begin{abstract}
This document summarizes the implementation of a secondary objective
	function for velocity kinematics on Baxter.
First, the jacobian is computed at the arm's current location.
Because the arm is assumed to start at a nonsingular location, the jacobian,
	which is a 6x7 matrix, has a one dimensional kernel, and therefore
	the solutions to $\xi = J \dot{q}$ can be written $a + b t$,
	for $a, b \in \mathbb{R}^7$ and $t \in \mathbb{R}$.
The secondary objective function attempts to maximize manipulability
	by choosing $t$ to maximize the cosine of the angle between 
	$\dot{q}$ and the gradient of manipulability.
So that the joint motions are not drastic, we limit the total joint velocities
	$\dot{q}$ to a maximum norm.
This secondary objective function maximizes manipulability, and tends to place
	the elbows at right angles.
It does not explicitly avoid exceeding the joint angles, however it will tend
	to avoid singularities.
\end{abstract}

\section{Solving for Joint Velocities}

In this section, we explicitly solve for the joint velocities $\dot{q}$ which
	achieve a desired end-effector twist $\xi$.

\subsection{Definition and Properties of the Jacobian}

The velocities are related by the jacobian matrix, which is a linearization
	of the velocity kinematics.
The jacobian matrix is defined by the following equation:

\begin{align}
	\xi & = J(q) \dot{q} \label{eq-def-of-jacobian}
\end{align}

Note that since Baxter has 7 joints and the end-effector twist contains 6
	components, the jacobian matrix is 6x7.
Therefore, assuming that the jacobian has full rank, it has a one-dimensional
	kernel.
Any solution of (\ref{eq-def-of-jacobian}) for a fixed $\xi$ is therefore
	not unique.

\subsection{Using the Pseudoinverse to Solve (\ref{eq-def-of-jacobian}) for
	given $\xi$}

\subsection{Finding the Kernel of $J$}


\section{Finding the Gradient of Manipulability}

If $J(q)$ is the jacobian, the manipulability is defined by:

\begin{align}
	\mu(q) & = \sqrt{ \det \left( J(q) J(q)^T \right) }  \label{eq-def-of-mu}
\end{align}

To find the gradient of $\mu(q)$:

\begin{align*}
	\frac {\partial \mu(q)}{\partial q_i}
		& = \frac 12 \frac {\frac{\partial}{\partial q_i} \det \left( J J^T \right)}
			{ \sqrt{ \det \left( J J^T \right) }}\\
		& = \frac 12 \frac {\det \left( J J^T\right) 
			\tr \left( (J J^T)^{-1} \frac{\partial}{\partial q_i} (J J^T) \right)}
		{\mu(q)}\\
		& = \frac{\mu(q)}2 \tr \left( (J J^T)^{-1} \frac{\partial}{\partial q_i} (J J^T) \right)
\end{align*} 

We don't actually have to compute $\nabla \mu(q)$, however:
	since we want to maximize the cosine of the angle between $\nabla \mu$,
	and $a + b t$, as long as we know the sign of $\nabla \mu(q)$, its magnitude
	is irrelevant.
Therefore, we can define the auxiliary quantity $c$, which is:

\begin{align}
	c & = \tr \left( (J J^T)^{-1} \nabla (J J^T) \right) \label{eq-def-of-c}\\
	& = \frac{2 }{\mu(q)} \nabla \mu(q) \nonumber
\end{align}

Since $\mu(q)$ is always positive (it is the positive branch of a square root),
	the vector $c$ is therefore proportional to $\nabla \mu(q)$ and of the same
	sign.
Therefore, to maximize the cosine of the angle between $\dot{q}$ and $\nabla \mu(q)$,
	it suffices to choose $t$ such that the cosine of the angle between $a + b t$
	and $c$ is maximized.

\section{Finding the optimal value of $t$}

We have reduced the problem to choosing $t$ such that the cosine of the angle 
	between $a + b t$ and $c$ is maximized.
For vectors $x, y \in \mathbb{R}^n$, the cosine between them is given by:

\begin{align}
	\cos \theta & = \frac{x \cdot y}{|x| |y|} \label{eq-cosine}
\end{align}

Therefore, the cosine of the angle between $a + b t$ and $c$ is given by:

\begin{align*}
	\cos \theta & = \frac {( a + b t) \cdot c}{\sqrt{ (a + b t)^2} \sqrt{c^2}}
\end{align*}

To maximize this quantity, we differentiate by t, and set equal to zero:

\begin{align*}
	0 & = \frac{d}{dt} \left( 
		\frac { a \cdot c + b \cdot c t}{\sqrt{ (a + b t)^2} \sqrt{c^2}} \right) \\
\end{align*}	

To determine which joint velocities $\dot{q}$ solve the equation for
	the end-effector velocity $\xi$, $\xi = J \dot{q}$,
	we first compute the pseudoinverse of $J$ and find $J^+ \xi$,
	and then we compute the kernel of $J$.
Since $J$ is assumed to be nonsingular, and it has six rows and seven columns,
	its kernel is spanned by a single vector $a$.
Therefore, for any $t \in \mathbb{R}$, $\dot{q} = J^+ \xi + a$ solves the equation
	$\xi = J \dot q$.
The problem is now reduced to finding the optimal value for $t$.

The secondary objective function attempts to maximize the cosine of the 
	angle between the j

\end{document}
